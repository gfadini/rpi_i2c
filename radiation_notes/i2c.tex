\documentclass{article}
\title{I2C table}
\author{Gabriele Fadini}
\usepackage{placeins}

\begin{document}
\maketitle

\iffalse
                Cristiano Fontana
                Caen sys -> detector PI del progetto
                Riunione Enea waste management as decommissioning
                Sogin
                Nucleco
                Finocchiaro

                Marco Ripani

                Rivelatori prodotti commerciali o no
                Dragon
                Ansaldo -> droni UGV da terra
                Manipolatori e strumentazione

                Hot cell
                Deposito 
                Ambiente ostile
                Radiation harness (per elettricità)
                Non sanno cosa ci sia dentro alle centrali
                Teleguidato

                Droni per grandi pareti verticali e mappare camini, noccioli, mappatura volante di lunghezza variabile


                700g 

                Supporto in alluminio
                O plastica sa stl
                **Telaio**
                + Robusto e certificato
                + dji
                + Matrice 600 (s600 o s900)
                + Medesimo controllore

                Brevetto volo sotto 4kg

                Enac

                Prove simulando carico
                Cuscino gonfiabile
\fi

\section{DAC commands}

Given the following requirements, specifications of the sensors and communication:
\begin{itemize}
    \item Start - Initialization (parametri)
        \begin{itemize}
            \item regulate high voltage tension thresholds \textbf{int 16bit}
            \item threshold \textbf{int 16bit}
        \end{itemize}
       
    \item Configuration and thresholds
    \item Stop
    \item Shutdown
    \item Open save file file (on rpi sensor)
        \begin{itemize}
            \item File with timestamp ISO (8 char)
        \end{itemize}
    \item Close file 
    \item Check status \textbf{255bytes}
        \begin{itemize}
            \item State machine status and ok
        \end{itemize}


    \item Sync orologio
    \item Get spectrum
    \item Get counts
    \item Get count gamma
    \item Get count n
    \item Get id
\end{itemize}



\section{Register list}
The following register table Tab. \ref{tab:reg_tab} is proposed.

\begin{table}[h]
    \centering
    \begin{tabular}{c|c|c|c|c}
        \textbf{Name} & \textbf{Address} & \textbf{Width} & \textbf{Access} & \textbf{Description}\\
        DEVICEID & 0x00  &  8    & RW     &   Device ID\\
        COUNT & 0x01  &  16    & RW     &   Measured counts n\\
        GAMMA    & 0x01  &  16    & RW     &  Measured $\gamma$ \\
        ID       & 0x01 & 8 & RW & Identification\\
        SPCTR    & 0x01  &  16   & RW      & Spectrum\\
        CMD     & 0x05  &  8    & RW    &   Command\\
        CMD\_PARM & 0x05   &  256 & RW      & Command parameters\\
        STATUS     & 0x06  &  2040    & RW      &   Status register\\
        FILNM  &       &  256  & RW     &   Filname\\

    \end{tabular}
    \caption{Proposed register table}
    \label{tab:reg_tab}
\end{table}



For command in CMD we need to use are at least 8 bit are enough to encode them with some margin:
\begin{itemize}
    \item Start Acquisition
    \item Stop Acquisition
    \item Open File + filename
    \item Close File
    \item Start Identification
    \item Set HV +  values
    \item Set treshold + values
    \item Get spectrum
    \item Get counts all
    \item Get counts $\gamma$
    \item Get counts n
\end{itemize}

Some commands need also another field of parameters for which there is an additional set of addresses CMD\_PARM, the actual 
size of this address depends on the threshold number to be set.\\
SPCTR encodes the dominant species of the analysis. There's margin, after the analysis is performed to send back either the additional species
of isotopes or readings regarding their percentages, if needed. In this case, one should specify the number of bins and levels
to send in the readings increasing the size of this register.\\
ID is the Identification of the 
For FILNAM we use ASCII Latin-1, making sure to eliminate wildcards, for a total length of 32 chars, the register needs to be 32bytes.

\FloatBarrier

\section{Status check}
    This enables the check of multiple settings and the success or not of the previous command, the acquisiton and all the
    actual thresholds of the device.
    \begin{table}[h!]
        \centering
        \begin{tabular}{c|c|c}
            \textbf{Bit \#} & \textbf{Access} & \textbf{Description}\\
            0:8      & RW     & Last command ($2^{8}-1$ commands)\\
            9      & RW     & Last command status (0 success, 1 fail)\\
            10    & RW     & File Open Status (0 file open, 1 error)\\
            11   & RW     & Acquisition running (0 running, 1 not running)\\
            12:27 & RW     & High voltage setting \textbf{HV}  ($2^{16}$ levels)\\
            28:43   & RW   & Counts levels \textbf{N} ($2^{16}$ levels)\\
            44:59 & RW     & $1^{st}$ threshold (each one can be set on $2^{16}$ levels)\\
            60:75 & RW     & $2^{nd}$ threshold\\
            76:91 & RW     & $3^{rd}$ threshold\\
            other to 255 & - & Reserved $\leftarrow$ can be used for other tresholds or status checks\\
        \end{tabular}
        \caption{Status register}
        \label{tab:status}
    \end{table}

\end{document}